%%%%%%%%%%%%%%%%%%%%%%%%%%%%%%%%%%%%%%%%%%%%%%%%%%%%%%%%%%%%%%%%%%%%%%%%%%%%%%%%%%%%%%%%
% Criação de Fluxograma usando LaTeX
%
% Assunto: o programa advtxt que no início entrega duas escolhas de objeto na qual dependendo da escolha você para ou continua, se você seguir você chega na escolha de verbo que se você escolher o correto você ganha, mas se escolher o errado perde.
%
% Autores:
%      Giovana Mel
%      Victor Emanuel
%      Marcelo Douglas
%
% Coordenação:
%     Prof. Dr. Ruben Carlo Benante
%
% Data: 2024-04-25
%%%%%%%%%%%%%%%%%%%%%%%%%%%%%%%%%%%%%%%%%%%%%%%%%%%%%%%%%%%%%%%%%%%%%%%%%%%%%%%%%%%%%%%%


%%%%%%%%%%%%%%%%%%%%%%%%%%%%%%%%%%%%%%%%%%%%%%%%%%%%%%%%%%%%%%%%%%%%%%%%%%%%%%%%%%%%%%%%
% Para gerar o PDF use o comando make com o makefile configurado:
%
%    $ make ext-programa2-benante-sobrenome1-sobrenome2.pdf
%
% O conteúdo do makefile é composto dos 3 seguintes comandos que ficam assim automatizados:
%    $ pdflatex exN-fluxograma.tex -o exN-fluxograma.pdf
%    $ bibtex biblio
%    $ pdflatex exN-fluxograma.tex -o exN-fluxograma.pdf


%%%%%%%%%%%%%%%%%%%%%%%%%%%%%%%%%%%%%%%%%%%%%%%%%%%%%%%%%%%%%%%%%%%%%%%%%%%%%%%%%%%%%%%%
% preambulo %%%%%%%%%%%%%%%%%%%%%%%%%%%%%%%%%%%%%%%%%%%%%%%%%%%%%%%%%%%%%%%%%%%%%%%%%%%%
\documentclass[a4paper,12pt]{article} %twocolumn
\usepackage[left=2.5cm,right=2cm,top=2.5cm,bottom=2cm]{geometry}
\usepackage[utf8]{inputenc} % letras acentuadas
\usepackage[portuguese]{babel} % tradução de títulos
\usepackage[colorlinks]{hyperref}
\usepackage{tikz} % para adicionar fluxogramas
\usepackage{algorithm} % ambiente para índice de algoritmos
\usepackage{algpseudocode} % fonte e estilo do algoritmo
\usepackage{graphicx} % permite adicionar imagens
\usepackage{indentfirst} % indenta o primeiro parágrafo também
\usepackage{url} % permite adicionar links de URLs e emails
% \usepackage{natbib}
%[noend]

\DeclareUrlCommand\email{\urlstyle{mm}} % comando para email bonito
\floatname{algorithm}{Algoritmo} % tradução da palavra algoritimo no ambiente de índice

\usetikzlibrary{shapes.geometric, shapes.symbols,arrows} % ajuste do tikz para incluir formas e setas

%%%%%%%%%%%%%%%%%%%%%%%%%%%%%%%%%%%%%%%%%%%%%%%%%%%%%%%%%%%%%%%%%%%%%%%%%%%%%%%%%%%%%%%%
% capa %%%%%%%%%%%%%%%%%%%%%%%%%%%%%%%%%%%%%%%%%%%%%%%%%%%%%%%%%%%%%%%%%%%%%%%%%%%%%%%%%
\title{Fluxograma: ex11-fluxograma}
\author{Giovana Mel \\ Victor Emanuel \\ Marcelo Douglas}

\begin{document}

\maketitle

%%%%%%%%%%%%%%%%%%%%%%%%%%%%%%%%%%%%%%%%%%%%%%%%%%%%%%%%%%%%%%%%%%%%%%%%%%%%%%%%%%%%%%%%
% definicao dos blocos do fluxograma (tikz) %%%%%%%%%%%%%%%%%%%%%%%%%%%%%%%%%%%%%%%%%%%%

\tikzstyle{line} = [draw, -latex']
\tikzstyle{startend} = [draw, ellipse,fill=red!20, minimum height=2em, node distance=1.55cm]
\tikzstyle{print} = [tape, fill=blue!20, draw, draw=black, minimum width=3cm, minimum height=1.4cm, text width=4.5em, text centered, tape bend top=none, tape bend height=0.2cm, node distance=1.55cm]
\tikzstyle{input} = [trapezium, trapezium left angle=60, trapezium right angle=90, minimum width=3cm, minimum height=1cm, text centered, draw=black, fill=blue!30, node distance=1.95cm]
\tikzstyle{process} = [rectangle, minimum width=3cm, minimum height=1cm, text centered, draw=black, fill=orange!30, node distance=1.55cm]

\tikzstyle{block} = [rectangle, draw, fill=blue!20, text width=5em, text centered, rounded corners, minimum height=4em, node distance=1.55cm]
\tikzstyle{decisionb} = [diamond, draw, fill=blue!20, text width=4.5em, text badly centered, inner sep=0pt, node distance=1.55cm]
\tikzstyle{decision} = [diamond, minimum width=3cm, minimum height=1cm, text centered, draw=black, fill=green!30, node distance=2.25cm]
\tikzstyle{empty} = [circle, fill=white, minimum width=0.01mm, node distance=2.55cm]

%%%%%%%%%%%%%%%%%%%%%%%%%%%%%%%%%%%%%%%%%%%%%%%%%%%%%%%%%%%%%%%%%%%%%%%%%%%%%%%%%%%%%%%%
% resumo
O jogo de Halloween, tem como cenário uma mansão assombrada onde há um assassino querendo lhe matar, e você tem sempre duas opções para se proteger onde uma você continuará viva e outra você morrerá, na primeira opção temos o objeto que terá que  escolher entre  faca ou garfo, onde na escolha da faca você sobreviverá e no garfo morrerá, você só responde a segunda pergunta se responder a primeira corretamente, e a última pergunta é o verbo onde tem o verbo correr ou matar, onde o verbo corretor é matar e você vence a partida

%%%%%%%%%%%%%%%%%%%%%%%%%%%%%%%%%%%%%%%%%%%%%%%%%%%%%%%%%%%%%%%%%%%%%%%%%%%%%%%

\begin{abstract}

\textbf{Assunto:} Programa tal tal

% descrever em poucas palavras seu projeto aqui

programa advtxt que no início entrega duas escolhas de objeto na qual dependendo da escolha você para ou continua, se você seguir você chega na escolha de verbo que se você escolher o correto você ganha, mas se escolher o errado perde.. Neste artigo iremos apresentar o seu fluxograma completo
% e (opcionalmente) o seu algoritmo.

Após a modelagem do fluxograma e desenvolvimento da lógica de programação em algoritmo,
o programa será implementado na Linguagem de Programação \texttt{C}


\textbf{Local:} Escola Politécnica de Pernambuco - UPE/POLI

\textbf{Órgão Financiador:} N/A

\textbf{Caracterização:} Modelagem, Projeto e Implementação de Software em Linguagem \texttt{C}

% Este é o fim do resumo.

\end{abstract}


%%%%%%%%%%%%%%%%%%%%%%%%%%%%%%%%%%%%%%%%%%%%%%%%%%%%%%%%%%%%%%%%%%%%%%%%%%%%%%%%%%%%%%%%
% artigo %%%%%%%%%%%%%%%%%%%%%%%%%%%%%%%%%%%%%%%%%%%%%%%%%%%%%%%%%%%%%%%%%%%%%%%%%%%%%%%
% seção de introdução %%%%%%%%%%%%%%%%%%%%%%%%%%%%%%%%%%%%%%%%%%%%%%%%%%%%%%%%%%%%%%%%%%
\section{Introdução}

% Descrever melhor seu projeto aqui

Este programa faz isso, isso e \texttt{também em fonte mono-espaçada faz isso}.

O programa será modelado em \textit{fluxograma} em uma primeira fase, em seguida
sua lógica será desenvolvida em formato de \textit{algoritmo}, para então
na terceira fase ser implementado em Linguagem de Programação \texttt{C}.

Exemplo de letra em \textit{itálico é com textit}, e em \textbf{bold face é com textbf} e \texttt{tal tal tal é mono-espaçada type-writer}.

%%%%%%%%%%%%%%%%%%%%%%%%%%%%%%%%%%%%%%%%%%%%%%%%%%%%%%%%%%%%%%%%%%%%%%%%%%%%%%%%%%%%%%%%
% seção de objetivos %%%%%%%%%%%%%%%%%%%%%%%%%%%%%%%%%%%%%%%%%%%%%%%%%%%%%%%%%%%%%%%%%%%
\section{Fluxograma}

% adicionar aqui o fluxograma

\begin{tikzpicture}
    % colocar nodos
    \node (inicio) [startend] {Inicio};
    \node (txta) [print, below of=inicio] {Texto A};
    \node (inp1) [input, below of=txta] {objeto};
    \node (tot1) [process, below of=inp1] {total $\leftarrow$ objeto + 1};
    \node (maior) [decision, below of=tot1] {total $>$ M};
    \node (sim) [print, right of=maior, node distance=4cm] {Total estourou};
    \node (nao) [print, below of=maior, node distance=2.4cm] {Total ok};
    \node (fim) [startend, below of=nao] {Fim};
    % \node (vazio1) [empty, right of=fim, node distance=4cm] {};
    % Desenhar as setas
    \path [line] (inicio) -- (txta);
    \path [line] (txta) -- (inp1);
    \path [line] (inp1) -- (tot1);
    \path [line] (tot1) -- (maior);
    \path [line] (maior) -- (sim);
    \path [line] (sim) -- (vazio1) -- (fim);
    \path [line] (maior) -- (nao);
    \path [line] (nao) -- (fim);
\end{tikzpicture}



\clearpage % inicia próxima seção em nova página
%%%%%%%%%%%%%%%%%%%%%%%%%%%%%%%%%%%%%%%%%%%%%%%%%%%%%%%%%%%%%%%%%%%%%%%%%%%%%%%%%%%%%%%%
% seção de justificativa %%%%%%%%%%%%%%%%%%%%%%%%%%%%%%%%%%%%%%%%%%%%%%%%%%%%%%%%%%%%%%%
% \section{Algoritmo}

% adicionar aqui o algoritmo (opcional)


% \clearpage % inicia próxima seção em nova página
%%%%%%%%%%%%%%%%%%%%%%%%%%%%%%%%%%%%%%%%%%%%%%%%%%%%%%%%%%%%%%%%%%%%%%%%%%%%%%%%%%%%%%%%
% Autores %%%%%%%%%%%%%%%%%%%%%%%%%%%%%%%%%%%%%%%%%%%%%%%%%%%%%%%%%%%%%%%%%%%%%%%%%%%%%%
\section*{Detalhamento dos Autores}

%%%%%%%%%%%%%%%%%%%%%%%%%%%%%%%%%%%%%%%%%%%%%%%%%%%%%%%%%%%%%%%%%%%%%%%%%%%%%%%%%%%%%%%%
% Discentes %%%%%%%%%%%%%%%%%%%%%%%%%%%%%%%%%%%%%%%%%%%%%%%%%%%%%%%%%%%%%%%%%%%%%%%%%%%%
\subsection*{Discentes}

\begin{enumerate}
    \item \textbf{Nome Completo:} Fulano de Tal Um
    \begin{description}
        \item [Email:] \email{blabla@poli.br}
        \item [Endereço:]
        \item [Matrícula:]
        \item [CPF:]
        \item [RG:]
        \item [Telefone:]
        \item [Currículo Lattes:] \url{http://lattes.cnpq.br/nnnnn}
    \end{description}

    \item \textbf{Nome Completo:} Fulano de Qual Dois
    \begin{description}
        \item [Email:] \email{blabla@poli.br}
        \item [Endereço:]
        \item [Matrícula:]
        \item [CPF:]
        \item [RG:]
        \item [Telefone:]
        \item [Currículo Lattes:] \url{http://lattes.cnpq.br/nnnnn}
    \end{description}

    \item \textbf{Nome Completo:} Fulano de Como Três
    \begin{description}
        \item [Email:] \email{blabla@poli.br}
        \item [Endereço:]
        \item [Matrícula:]
        \item [CPF:]
        \item [RG:]
        \item [Telefone:]
        \item [Currículo Lattes:] \url{http://lattes.cnpq.br/nnnnn}
    \end{description}

    % \item \textbf{Nome Completo:} Fulano de tal
    % \begin{description}
        % \item [Email:] \email{blabla@poli.br}
        % \item [Endereço:]
        % \item [Matrícula:]
        % \item [CPF:]
        % \item [RG:]
        % \item [Telefone:]
        % \item [Currículo Lattes:] \url{http://lattes.cnpq.br/nnnnn}
    % \end{description}

%     \item \textbf{Nome Completo:} Fulano de tal
%     \begin{description}
%         \item [Email:] \email{blabla@poli.br}
%         \item [Endereço:]
%         \item [Matrícula:]
%         \item [CPF:]
%         \item [RG:]
%         \item [Telefone:]
%         \item [Currículo Lattes:] \url{http://lattes.cnpq.br/nnnnn}
%     \end{description}
\end{enumerate}


%%%%%%%%%%%%%%%%%%%%%%%%%%%%%%%%%%%%%%%%%%%%%%%%%%%%%%%%%%%%%%%%%%%%%%%%%%%%%%%%%%%%%%%%
% Docentes %%%%%%%%%%%%%%%%%%%%%%%%%%%%%%%%%%%%%%%%%%%%%%%%%%%%%%%%%%%%%%%%%%%%%%%%%%%%%
\subsection*{Docentes}

\begin{enumerate}
    \item \textbf{Nome Completo:} Ruben Carlo Benante
    \begin{description}
        \item [Email:] \email{rcb@upe.br}
        \item [Matrícula:] 11238-0
        \item [Currículo Lattes:] \url{http://lattes.cnpq.br/3366717378277623}
    \end{description}
\end{enumerate}


%%%%%%%%%%%%%%%%%%%%%%%%%%%%%%%%%%%%%%%%%%%%%%%%%%%%%%%%%%%%%%%%%%%%%%%%%%%%%%%%%%%%%%%%
% referências bibliográficas %%%%%%%%%%%%%%%%%%%%%%%%%%%%%%%%%%%%%%%%%%%%%%%%%%%%%%%%%%%
%\section*{Referências Bibliográficas}

% cite todos, mesmo os não referenciados %%%%%%%%%%%%%%%%%%%%%%%%%%%%%%%%%%%%%%%%%%%%%%%
\nocite{*}


%%%%%%%%%%%%%%%%%%%%%%%%%%%%%%%%%%%%%%%%%%%%%%%%%%%%%%%%%%%%%%%%%%%%%%%%%%%%%%%%%%%%%%%%
% se necessario %%%%%%%%%%%%%%%%%%%%%%%%%%%%%%%%%%%%%%%%%%%%%%%%%%%%%%
% troca autor and autor por autor & autor, na bibliografia. O dcu usa "and"
%\renewcommand{\harvardand}{\&} % troca and pro &. O dcu usa "and"

% Estilos de bibliografia %%%%%%%%%%%%%%%%%%%%%%%%%%%%%%%%%%%%%%%%%%%%%%%%%%%%%%%%%%%%%%
% \bibliographystyle{abnt-alf} % Estilo alfabético da ABNT. Opção [num] para estilo numérico
% \bibliographystyle{apalike}
% \bibliographystyle{dcu} %citacao como (autor and autor, ano). Parece apalike. Rev. Control. Automacao. Use com harvard
% \bibliographystyle{agsm} % padrao harvard fica (autor & autor ano).
\bibliographystyle{acm}

%%%%%%%%%%%%%%%%%%%%%%%%%%%%%%%%%%%%%%%%%%%%%%%%%%%%%%%%%%%%%%%%%%%%%%%%%%%%%%%%%%%%%%%%
% arquivo de banco de dados das referências %%%%%%%%%%%%%%%%%%%%%%%%%%%%%%%%%%%%%%%%%%%%
% renomear para o número do exercício correto
% o arquivo de bibliografia pode se chamar qualquer coisa, isso não muda o comando de gerar o PDF.
% Por exemplo para 'mybiblio.bib', use \bibliography{mybiblio} e os comandos pdflatex e bibtex continuam os mesmos identicos com exN.
\bibliography{biblio}

\end{document}
