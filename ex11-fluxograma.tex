%%%%%%%%%%%%%%%%%%%%%%%%%%%%%%%%%%%%%%%%%%%%%%%%%%%%%%%%%%%%%%%%%%%%%%%%%%%%%%%%%%%%%%%%
% Criação de Fluxograma usando LaTeX
%
% Assunto: o programa advtxt que no início entrega duas escolhas de objeto na qual dependendo da escolha você para ou continua, se você seguir você chega na escolha de verbo que se você escolher o correto você ganha, mas se escolher o errado perde.
%
% Autores:
%      Giovana Mel
%      Victor Emanuel
%      Marcelo Douglas
%
% Coordenação:
%     Prof. Dr. Ruben Carlo Benante
%
% Data: 2024-04-25
%%%%%%%%%%%%%%%%%%%%%%%%%%%%%%%%%%%%%%%%%%%%%%%%%%%%%%%%%%%%%%%%%%%%%%%%%%%%%%%%%%%%%%%%


%%%%%%%%%%%%%%%%%%%%%%%%%%%%%%%%%%%%%%%%%%%%%%%%%%%%%%%%%%%%%%%%%%%%%%%%%%%%%%%%%%%%%%%%
% Para gerar o PDF use o comando make com o makefile configurado:
%
%    $ make ext-programa2-benante-sobrenome1-sobrenome2.pdf
%
% O conteúdo do makefile é composto dos 3 seguintes comandos que ficam assim automatizados:
%    $ pdflatex exN-fluxograma.tex -o exN-fluxograma.pdf
%    $ bibtex biblio
%    $ pdflatex exN-fluxograma.tex -o exN-fluxograma.pdf


%%%%%%%%%%%%%%%%%%%%%%%%%%%%%%%%%%%%%%%%%%%%%%%%%%%%%%%%%%%%%%%%%%%%%%%%%%%%%%%%%%%%%%%%
% preambulo %%%%%%%%%%%%%%%%%%%%%%%%%%%%%%%%%%%%%%%%%%%%%%%%%%%%%%%%%%%%%%%%%%%%%%%%%%%%
\documentclass[a4paper,12pt]{article} %twocolumn
\usepackage[left=2.5cm,right=2cm,top=2.5cm,bottom=2cm]{geometry}
\usepackage[utf8]{inputenc} % letras acentuadas
\usepackage[portuguese]{babel} % tradução de títulos
\usepackage[colorlinks]{hyperref}
\usepackage{tikz} % para adicionar fluxogramas
\usepackage{algorithm} % ambiente para índice de algoritmos
\usepackage{algpseudocode} % fonte e estilo do algoritmo
\usepackage{graphicx} % permite adicionar imagens
\usepackage{indentfirst} % indenta o primeiro parágrafo também
\usepackage{url} % permite adicionar links de URLs e emails
% \usepackage{natbib}
%[noend]

\DeclareUrlCommand\email{\urlstyle{mm}} % comando para email bonito
\floatname{algorithm}{Algoritmo} % tradução da palavra algoritimo no ambiente de índice

\usetikzlibrary{shapes.geometric, shapes.symbols,arrows} % ajuste do tikz para incluir formas e setas

%%%%%%%%%%%%%%%%%%%%%%%%%%%%%%%%%%%%%%%%%%%%%%%%%%%%%%%%%%%%%%%%%%%%%%%%%%%%%%%%%%%%%%%%
% capa %%%%%%%%%%%%%%%%%%%%%%%%%%%%%%%%%%%%%%%%%%%%%%%%%%%%%%%%%%%%%%%%%%%%%%%%%%%%%%%%%
\title{Fluxograma: ex11-fluxograma}
\author{Giovana Mel \\ Victor Emanuel \\ Marcelo Douglas}

\begin{document}

\maketitle

%%%%%%%%%%%%%%%%%%%%%%%%%%%%%%%%%%%%%%%%%%%%%%%%%%%%%%%%%%%%%%%%%%%%%%%%%%%%%%%%%%%%%%%%
% definicao dos blocos do fluxograma (tikz) %%%%%%%%%%%%%%%%%%%%%%%%%%%%%%%%%%%%%%%%%%%%

\tikzstyle{line} = [draw, -latex']
\tikzstyle{startend} = [draw, ellipse,fill=red!20, minimum height=2em, node distance=1.55cm]
\tikzstyle{print} = [tape, fill=blue!20, draw, draw=black, minimum width=3cm, minimum height=1.4cm, text width=4.5em, text centered, tape bend top=none, tape bend height=0.2cm, node distance=1.55cm]
\tikzstyle{input} = [trapezium, trapezium left angle=60, trapezium right angle=90, minimum width=3cm, minimum height=1cm, text centered, draw=black, fill=blue!30, node distance=1.95cm]
\tikzstyle{process} = [rectangle, minimum width=3cm, minimum height=1cm, text centered, draw=black, fill=orange!30, node distance=1.55cm]

\tikzstyle{block} = [rectangle, draw, fill=blue!20, text width=5em, text centered, rounded corners, minimum height=4em, node distance=1.55cm]
\tikzstyle{decisionb} = [diamond, draw, fill=blue!20, text width=4.5em, text badly centered, inner sep=0pt, node distance=1.55cm]
\tikzstyle{decision} = [diamond, minimum width=3cm, minimum height=1cm, text centered, draw=black, fill=green!30, node distance=2.25cm]
\tikzstyle{empty} = [circle, fill=white, minimum width=0.01mm, node distance=2.55cm]

%%%%%%%%%%%%%%%%%%%%%%%%%%%%%%%%%%%%%%%%%%%%%%%%%%%%%%%%%%%%%%%%%%%%%%%%%%%%%%%%%%%%%%%%
% resumo
O jogo de Halloween, tem como cenário uma mansão assombrada onde há um assassino querendo lhe matar, e você tem sempre duas opções para se proteger onde uma você continuará viva e outra você morrerá, na primeira opção temos o objeto que terá que  escolher entre  faca ou garfo, onde na escolha da faca você sobreviverá e no garfo morrerá, você só responde a segunda pergunta se responder a primeira corretamente, e a última pergunta é o verbo onde tem o verbo correr ou matar, onde o verbo corretor é matar e você vence a partida

%%%%%%%%%%%%%%%%%%%%%%%%%%%%%%%%%%%%%%%%%%%%%%%%%%%%%%%%%%%%%%%%%%%%%%%%%%%%%%%

\begin{abstract}

\textbf{Assunto:} Programa tal tal

% descrever em poucas palavras seu projeto aqui

programa advtxt que no início entrega duas escolhas de objeto na qual dependendo da escolha você para ou continua, se você seguir você chega na escolha de verbo que se você escolher o correto você ganha, mas se escolher o errado perde.. Neste artigo iremos apresentar o seu fluxograma completo
% e (opcionalmente) o seu algoritmo.

Após a modelagem do fluxograma e desenvolvimento da lógica de programação em algoritmo,
o programa será implementado na Linguagem de Programação \texttt{C}


\textbf{Local:} Escola Politécnica de Pernambuco - UPE/POLI

\textbf{Órgão Financiador:} N/A

\textbf{Caracterização:} Modelagem, Projeto e Implementação de Software em Linguagem \texttt{C}

% Este é o fim do resumo.

\end{abstract}


%%%%%%%%%%%%%%%%%%%%%%%%%%%%%%%%%%%%%%%%%%%%%%%%%%%%%%%%%%%%%%%%%%%%%%%%%%%%%%%%%%%%%%%%
% artigo %%%%%%%%%%%%%%%%%%%%%%%%%%%%%%%%%%%%%%%%%%%%%%%%%%%%%%%%%%%%%%%%%%%%%%%%%%%%%%%
% seção de introdução %%%%%%%%%%%%%%%%%%%%%%%%%%%%%%%%%%%%%%%%%%%%%%%%%%%%%%%%%%%%%%%%%%
\section{Introdução}

% Descrever melhor seu projeto aqui

O jogo de Halloween, tem como cenário uma mansão assombrada onde há um assassino
querendo lhe matar, e você tem sempre duas opções para se proteger onde uma você
continuará viva e outra você morrerá, na primeira opção temos o objeto que terá que
escolher entre faca ou garfo, onde na escolha da faca você sobreviverá e no garfo
morrerá, você só responde a segunda pergunta se responder a primeira corretamente, e a
última pergunta é o verbo onde tem o verbo correr ou matar, onde o verbo corretor é
matar e você vence a partida
%%%%%%%%%%%%%%%%%%%%%%%%%%%%%%%%%%%%%%%%%%%%%%%%%%%%%%%%%%%%%%%%%%%%%%%%%%%%%%%%%%%%%%%%
% seção de objetivos %%%%%%%%%%%%%%%%%%%%%%%%%%%%%%%%%%%%%%%%%%%%%%%%%%%%%%%%%%%%%%%%%%%
\section{Fluxograma}

% adicionar aqui o fluxograma

\begin{tikzpicture}
    % colocar nodos
    \node (inicio) [startend] {Inicio};
    \node (txta) [print, below of=inicio] {Texto A};
    \node (inp1) [input, below of=txta] {objeto};
    \node (tot1) [process, below of=inp1] {total $\leftarrow$ objeto + 1};
    \node (maior) [decision, below of=tot1] {total $>$ M};
    \node (sim) [print, right of=maior, node distance=4cm] {Total estourou};
    \node (nao) [print, below of=maior, node distance=2.4cm] {Total ok};
    \node (fim) [startend, below of=nao] {Fim};
    % \node (vazio1) [empty, right of=fim, node distance=4cm] {};
    % Desenhar as setas
    \path [line] (inicio) -- (txta);
    \path [line] (txta) -- (inp1);
    \path [line] (inp1) -- (tot1);
    \path [line] (tot1) -- (maior);
    \path [line] (maior) -- (sim);
    \path [line] (sim) -- (vazio1) -- (fim);
    \path [line] (maior) -- (nao);
    \path [line] (nao) -- (fim);
\end{tikzpicture}



\clearpage % inicia próxima seção em nova página
%%%%%%%%%%%%%%%%%%%%%%%%%%%%%%%%%%%%%%%%%%%%%%%%%%%%%%%%%%%%%%%%%%%%%%%%%%%%%%%%%%%%%%%%
% seção de justificativa %%%%%%%%%%%%%%%%%%%%%%%%%%%%%%%%%%%%%%%%%%%%%%%%%%%%%%%%%%%%%%%
% \section{Algoritmo}

% adicionar aqui o algoritmo (opcional)

algoritmo Jogo_advtxt;
variáveis
    verbo : literal;
    objeto : literal;
fim-variáveis

início
    imprima("Bem-vindo ao jogo de Halloween! Você está em uma mansão assombrada e há um assassino querendo te matar. Você terá que escolher uma resposta entre duas opções; se escolher errado, morrerá. Se escolher certo, você chegará ao fim!\n");
    imprima("Você está na mansão, e para se defender, você tem duas opções: escolha (faca ou garfo):\n");
    objeto := leia();

    se objeto = "faca" então
        imprima("Você escolheu a faca. Você pega a faca para se defender e pode continuar o jogo.\n");
        imprima("Você está a um passo de acabar. Vamos decidir o que irá fazer: você correr ou matar o assassino?\n");
        verbo := leia();

        se verbo = "matar" então
            imprima("Você escolheu matar. Você consegue matar o assassino. Parabéns, você ganhou!!!!!\n");
        senão
            imprima("Você escolheu correr e o assassino pegou você. Você morreu.\n");
        fim-se

    senão se objeto = "garfo" então
        imprima("Você escolheu o garfo. Tentou se defender e morreu. Tente novamente.\n");
    senão
        imprima("Opção inválida. Tente novamente.\n");
    fim-se
    fim-se
fim



% \clearpage % inicia próxima seção em nova página
%%%%%%%%%%%%%%%%%%%%%%%%%%%%%%%%%%%%%%%%%%%%%%%%%%%%%%%%%%%%%%%%%%%%%%%%%%%%%%%%%%%%%%%%
% Autores %%%%%%%%%%%%%%%%%%%%%%%%%%%%%%%%%%%%%%%%%%%%%%%%%%%%%%%%%%%%%%%%%%%%%%%%%%%%%%
\section*{Detalhamento dos Autores}

%%%%%%%%%%%%%%%%%%%%%%%%%%%%%%%%%%%%%%%%%%%%%%%%%%%%%%%%%%%%%%%%%%%%%%%%%%%%%%%%%%%%%%%%
% Discentes %%%%%%%%%%%%%%%%%%%%%%%%%%%%%%%%%%%%%%%%%%%%%%%%%%%%%%%%%%%%%%%%%%%%%%%%%%%%
\subsection*{Discentes}

\begin{enumerate}
    \item \textbf{Nome Completo:} Giovana Mel ivo de Oliveira. AUTOR 1
    \begin{description}
        \item [Email:] \email{gmio@poli.br}
        \item [Endereço: Rua Manoel Pessoa de Luna filho, 73, Prado]
        \item [Matrícula:300889892]
        \item [CPF: 120.307.334-82]
        \item [RG:10.303.015]
        \item [Telefone:(81) 9 9983-2287]
        \end{description}

    \item \textbf{Nome Completo:} Victor Emanuel de Souza Silva Ramos. AUTOR 2
        \begin{description}
        \item [Email:] \email{vessr@poli.br}
        \item [Endereço:: Rua Fausto Cardoso, 59, Madalena]
        \item [Matrícula:300884225]
        \item [CPF:147.269.684-08]
        \item [RG:10.658.232]
        \item [Telefone: (81) 9 8275-9092]
        \end{description}

    \item \textbf{Nome Completo:} Marcelo Douglas Rodrigues. AUTOR 3
        \begin{description}
        \item [Email:] \email{mdr@poli.br}
        \item [Endereço:: Ouro Preto - Olinda]
        \item [Telefone:(81) 9 8417-0538]
        \end{description}

   %%%%%%%%%%%%%%%%%%%%%%%%%%%%%%%%%%%%%%%%%%%%%%%%%%%%%%%%%%%%%%%%%%%%%%%%%%%%%%%%%%%%%%%%
% Docentes %%%%%%%%%%%%%%%%%%%%%%%%%%%%%%%%%%%%%%%%%%%%%%%%%%%%%%%%%%%%%%%%%%%%%%%%%%%%%
\subsection*{Docentes}

\begin{enumerate}
    \item \textbf{Nome Completo:} Ruben Carlo Benante
    \begin{description}
        \item [Email:] \email{rcb@upe.br}
        \item [Matrícula:] 11238-0
        \item [Currículo Lattes:] \url{http://lattes.cnpq.br/3366717378277623}
    \end{description}
\end{enumerate}


%%%%%%%%%%%%%%%%%%%%%%%%%%%%%%%%%%%%%%%%%%%%%%%%%%%%%%%%%%%%%%%%%%%%%%%%%%%%%%%%%%%%%%%%
% referências bibliográficas %%%%%%%%%%%%%%%%%%%%%%%%%%%%%%%%%%%%%%%%%%%%%%%%%%%%%%%%%%%
%\section*{Referências Bibliográficas}
Manual G-portugol, Thiago Silva
% cite todos, mesmo os não referenciados %%%%%%%%%%%%%%%%%%%%%%%%%%%%%%%%%%%%%%%%%%%%%%%
\nocite{*}
Manual G-portugol, Thiago Silva
Rubens C.B. arquivos modelos e exemplos

\bibliography{biblio}

\end{document}
